\documentclass[11pt]{article}
\usepackage{amsmath}
\usepackage{graphicx}
\usepackage{hyperref}
\usepackage[utf8]{inputenc}

\title{Synthese - IEEE754}
\author{Jaunart Gilles}
\date{}

\begin{document}
\maketitle

\section{Paramètres et valeurs speciales:}

Lettre majuscule représente la taille \\
Lettre minuscule représente la valeur \\
\\
E = exposant \\
M = mantisse \\
B = biais \\
S = bit de signe

\subsection{Paramètres IEEE754:}
\begin{tabular}{|c|c|c|c|c|}
\hline
\textbf{Précision} & \textbf{Taille (bits)} & \textbf{E} & \textbf{M} & \textbf{B} \\ \hline
Simple             & 32                     & 8          & 23         & 127        \\ \hline
Double             & 64                     & 11         & 52         & 1023       \\ \hline
\end{tabular}

\subsection{Valeurs speciales:}
\begin{tabular}{|c|c|c|c|}
\hline
\textbf{s} & \textbf{e} & \textbf{m} & \textbf{valeur} \\ \hline
0          & $2^{E-1}$  & 0          & $+ \infty$      \\ \hline
1          & $2^{E-1}$  & 0          & $- \infty$      \\ \hline
0 / 1      & $2^{E-1}$  & $\neq$0    & NaN             \\ \hline
\end{tabular}

\newpage

\section{Conversion decimal vers IEEE754}

\subsection{Signe:}
Si $x > 0$, le bit de signe vaut 0. Si $x < 0$, le bit de signe vaut 1. \\
Par la suite on prendra la valeur absolue de $X$.

\subsection{Normalisation:}
$x$ est normalisé à l’aide de $n$ divisions successives afin que $x'.2^n=x$ où $1 \leq x' <2$. \\
Par exemple, $x=6.2$ est divisé $n=2$ fois par $2$ pour obtenir $x'=1,55$ tel que $1,55.2^2=6,2$

\subsection{Déduction de l'exposant:}
On pose $2^n=2^{e-B}$, ce qui permet de déduire $e=n+B$. On vérifie alors que le nombre est bien représentable. \\
Si $0<e<2^{E-1}$, alors la représentation normalisée doit etre utilisée. \\
Si $e \leq 0$, alors on doit utiliser la dénormalisée.

\subsection{Déduction de la mantisse en représentation normalisée:}
On pose $x'=1+\frac{m}{2^M}$, ce qui permet de déduire $m=(x'-1).2^M$

\subsection{Déduction de la mantisse en représenation dénormalisée:}
Ici l'exposant est fixe, $e=1$. On pose $x=\frac{m}{2^M}.2^{1-B}$, ce qui permet de déduire $m=x.2^{M-(1-B)}$

\newpage

\section{Arrondi avec IEEE754}

\subsection{Round-to-nearest-even:}
On arrondit au nombre le plus proche. Ex: $2,4 \rightarrow 2$ ; $1,7 \rightarrow 2$ \\
Si on se retrouve pile entre 2 nombres pairs (ex: $1,35$ ; $2,65$ ; $3,15$) alors on arrondit au nombre pair le plus proche. Ex: $1,35 \rightarrow 1,4$ car $4$ est pair et pas $3$ ; 
$2,65 \rightarrow 2,6$ ; $3,15 \rightarrow 3,2$

\subsection{Round-toward-zero:}
On tronque la partie non-représentable. Ex: $1,4 \rightarrow 1$ ; $2,7 \rightarrow 2$ ; $-2,5 \rightarrow -2$

\subsection{Round-down:}
Il s'agit de l'arrondi par défaut. Ex: $1,4 \rightarrow 1$ ; $2,7 \rightarrow 2$ ; $-2,5 \rightarrow -3$

\subsection{Round-up:}
Il s'agit de l'arrondi par excés. Ex: $1,4 \rightarrow 2$ ; $2,7 \rightarrow 3$ ; $-2,5 \rightarrow -2$

\newpage

\section{Typologie des erreurs}

\subsection{Erreur vraie:}
$\Delta x = X - \^X$ où $\^X$ est la valeur de nombre représenté.

\subsection{Erreur absolue:}
$|\Delta x| = |X - \^X|$

\subsection{Erreur relative:}
$\epsilon_X = \frac{|X-\^X|}{|X|}$ \\

\subsection{Epsilon machine:}
La précision machine ou "epsilon machine" est une borne sur l'erreur relative qui 
dépend du format de représentation, en particulier de la taille de la mantisse $M$. \\
\begin{center}
    $\frac{|X-\^X|}{|X|} \leq \epsilon_M$ où $\epsilon_M = 2^{-(M+1)}$
\end{center}

\end{document}
